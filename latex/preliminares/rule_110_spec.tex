\section{\texttt{Rule 110}}

A \texttt{Rule 110} é uma regra simples de automato celular
unidimensional pertencente à classe dos
\textit{elementary cellular automata}.
Esses automatos consistem em uma fita infinita de células
dispostas em uma linha, onde cada célula pode estar em um de
dois estados: ativo (\texttt{1}) ou inativo (\texttt{0}).
O estado de cada célula em um instante \( t+1 \) é determinado
pelo estado dela e de suas duas vizinhas em \( t \), conforme
uma regra fixa.

Formalmente, a evolução da Rule 110 é definida pela função de
transição local que mapeia o estado de uma célula e seus
vizinhos para o estado seguinte. Definindo:
\begin{equation*}
A = \{(1,1,0), (1,0,1),(0,1,1), (0,1,0), (0,0,1)\}
\end{equation*}

\index{função de transição}
\noindent Podemos descrever essa função de transição como:

\[
f(a, b, c) =
\begin{cases}
1, & \text{se } (a, b, c) \in A, \\
0, & \text{caso contrário}.
\end{cases}
\]

Aqui, \( a \), \( b \), \( c \) representam o estado da célula
da esquerda, a célula atual e a célula da direita,
respectivamente.

Essa regra pode ser representada compactamente na seguinte
tabela de transição, onde as entradas representam
\( (a, b, c) \), e as saídas o estado resultante:

\[
\begin{array}{c|c}
\text{Configuração } (a, b, c) & \text{Próximo estado} \\
\hline
111 & 0 \\
110 & 1 \\
101 & 1 \\
100 & 0 \\
011 & 1 \\
010 & 1 \\
001 & 1 \\
000 & 0 \\
\end{array}
\]

Começando com uma única célula ativa, e representando
uma geração por linha, podemos visualizar a evolução
do autômato da seguinte forma:
\index{autômato}

\begin{verbatim}
            O 
           OO 
          OOO 
         OO O 
        OOOOO 
       OO   O 
      OOO  OO 
     OO O OOO
\end{verbatim}

Onde espaços representam células inativas e \verb|O|
representam células ativas.