\section{Implementação da \texttt{Rule 110}}

Para implementar \texttt{Rule 110} em Python,
colocamos as regras em um mapa, e usamos um procedimento
\verb|generation| para calcular cada geração. O procedimento
\verb|rule| aplica a regra em uma única célula usando
o mapa de regras.
\index{geração}

\begin{lstlisting}
def generation(prev, curr):
    i = 1
    while i < len(prev)-1:
        curr[i] = rule(prev, i)
        i += 1
    return curr

rulemap = {
    "   ": " ",
    "O  ": " ",
    " O ": "O",
    "  O": "O",
    "OO ": "O",
    "O O": "O",
    " OO": "O",
    "OOO": " ",
}

def rule(prev, i):
    chunk = make_str(prev[i-1:i+2])
    return rulemap[chunk]
\end{lstlisting}

Duas funções utilitárias foram criadas para
mapear strings em listas de caracteres e vice-versa.

\begin{lstlisting}
def make_array(str):
    out = []
    i = 0
    while i < len(str):
        out += [str[i]]
        i += 1
    return out

def make_str(chunk):
    out = ""
    i = 0
    while i < len(chunk):
        out += chunk[i]
        i += 1
    return out

\end{lstlisting}
\clearpage
\index{double buffering}
Por fim, utilizamos uma técnica chamada
\textit{double buffering} para calcular e imprimir na tela as
gerações. Calculamos apenas \(32\) gerações.

\begin{lstlisting}
A = make_array("                                O ")
B = make_array("                                  ")
i = 0
while i < 32:
    print(make_str(A))
    B = generation(A, B)

    C = A
    A = B
    B = C

    i += 1
\end{lstlisting}

Podemos usar \texttt{CPython} para rodar esse arquivo,
e isso prova que Python é TC. Deixaremos como exercício
para o leitor a implementação de Python em \texttt{Rule 110}.