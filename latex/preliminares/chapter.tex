\chapter{Preliminares}

Antes da formalização da matemática em cima da teoria
de conjuntos, um conceito crucial que permanecia intuitivo
e informal era a ideia de uma "função computável".
Apesar disso, nos sistemas de lógica propostos por Hilbert,
um conjunto de axiomas era válido se, e somente se existisse
uma "função computável" para decidir se uma sentença
lógica era um axioma ou não (isto é, o conjunto de axiomas
tem que ser obrigatoriamente um \textit{conjunto computável})
\cite{godelIncompletenessHilbert} \cite{churchTuring}.

\index{Máquina de Turing}
Turing criou a Máquina de Turing em 1937 \cite{Turing1937}
justamente para formalizar a intuição de "função computável".
Ele não foi o único a tentar formalizar essa ideia:
Alonzo Church \cite{church1985calculi},
Emil Post \cite{Post1943},
Stephen Kleene \cite{kleeneMetamathematics},
Raymond Smullyan \cite{Smullyan1969}
e outros também desenvolveram
definições independentes.
Curiosamente, todas essas definições foram demonstradas
serem equivalentes, e foi conjecturado que tudo que é
computável, na noção intuitiva que era usada anteriormente,
era computável por uma máquina de Turing
\cite{churchTuring}. Com isso, dizemos que esses modelos
de computação são capazes de \textit{computação universal}.
\index{computação universal}

\section{Completude de Turing}

Dizemos que dois modelos de computação são equivalentes
quando é possível simular um dentro do outro. Quando
algum modelo é equivalente a máquinas de Turing, dizemos
que esse sistema possui completude de Turing, ou que
esse sistema é \textit{Turing Completo} (TC).
\index{Turing Completo} \index{TC}

Por exemplo,
em \textit{Universality in Elementary Cellular Automata}
\cite{Cook2004},
o autor mostra que é possível simular uma máquina
de Turing usando os sistemas de
Emil Post, e usando um sistema chamado \textit{Cyclic Tag System} é possível simular os sistemas de Emil Post,
e usando um sistema chamado \textit{Glider System}
é possível simular o \textit{Cyclic Tag System},
e por fim, ele mostra que é possível usar \texttt{Rule 110}
para simular o \textit{Glider System}. Com isso,
simulando \texttt{Rule 110} em uma máquina de Turing,
é possível mostrar que todos esses modelos de computação
são equivalentes.

\index{\texttt{Rule 110}}
A utilidade de saber que \texttt{Rule 110} é TC, é que esse sistema é incrivelmente
fácil de ser simulado, abrindo portas para verificar
que outros modelos de computação também são TC.
Esse é o primeiro objetivo deste trabalho.

Outra coisa interessante é que todo sistema TC é capaz de
simular a si mesmo. A recíproca não é verdade, existem
sistemas capazes de simular a si mesmo que não são TC.
A segunda parte desse trabalho (a parte mais
trabalhosa) tem como objetivo simular um subconjunto de Python
dentro desse mesmo subconjunto.
\section{\texttt{Rule 110}}

A \texttt{Rule 110} é uma regra simples de automato celular
unidimensional pertencente à classe dos
\textit{elementary cellular automata}.
Esses automatos consistem em uma fita infinita de células
dispostas em uma linha, onde cada célula pode estar em um de
dois estados: ativo (\texttt{1}) ou inativo (\texttt{0}).
O estado de cada célula em um instante \( t+1 \) é determinado
pelo estado dela e de suas duas vizinhas em \( t \), conforme
uma regra fixa.

Formalmente, a evolução da Rule 110 é definida pela função de
transição local que mapeia o estado de uma célula e seus
vizinhos para o estado seguinte. Definindo:
\begin{equation*}
A = \{(1,1,0), (1,0,1),(0,1,1), (0,1,0), (0,0,1)\}
\end{equation*}

\index{função de transição}
\noindent Podemos descrever essa função de transição como:

\[
f(a, b, c) =
\begin{cases}
1, & \text{se } (a, b, c) \in A, \\
0, & \text{caso contrário}.
\end{cases}
\]

Aqui, \( a \), \( b \), \( c \) representam o estado da célula
da esquerda, a célula atual e a célula da direita,
respectivamente.

Essa regra pode ser representada compactamente na seguinte
tabela de transição, onde as entradas representam
\( (a, b, c) \), e as saídas o estado resultante:

\[
\begin{array}{c|c}
\text{Configuração } (a, b, c) & \text{Próximo estado} \\
\hline
111 & 0 \\
110 & 1 \\
101 & 1 \\
100 & 0 \\
011 & 1 \\
010 & 1 \\
001 & 1 \\
000 & 0 \\
\end{array}
\]

Começando com uma única célula ativa, e representando
uma geração por linha, podemos visualizar a evolução
do autômato da seguinte forma:
\index{autômato}

\begin{verbatim}
            O 
           OO 
          OOO 
         OO O 
        OOOOO 
       OO   O 
      OOO  OO 
     OO O OOO
\end{verbatim}

Onde espaços representam células inativas e \verb|O|
representam células ativas.
\section{Implementação da \texttt{Rule 110}}

Para implementar \texttt{Rule 110} em Python,
colocamos as regras em um mapa, e usamos um procedimento
\verb|generation| para calcular cada geração. O procedimento
\verb|rule| aplica a regra em uma única célula usando
o mapa de regras.
\index{geração}

\begin{lstlisting}
def generation(prev, curr):
    i = 1
    while i < len(prev)-1:
        curr[i] = rule(prev, i)
        i += 1
    return curr

rulemap = {
    "   ": " ",
    "O  ": " ",
    " O ": "O",
    "  O": "O",
    "OO ": "O",
    "O O": "O",
    " OO": "O",
    "OOO": " ",
}

def rule(prev, i):
    chunk = make_str(prev[i-1:i+2])
    return rulemap[chunk]
\end{lstlisting}

Duas funções utilitárias foram criadas para
mapear strings em listas de caracteres e vice-versa.

\begin{lstlisting}
def make_array(str):
    out = []
    i = 0
    while i < len(str):
        out += [str[i]]
        i += 1
    return out

def make_str(chunk):
    out = ""
    i = 0
    while i < len(chunk):
        out += chunk[i]
        i += 1
    return out

\end{lstlisting}
\clearpage
\index{double buffering}
Por fim, utilizamos uma técnica chamada
\textit{double buffering} para calcular e imprimir na tela as
gerações. Calculamos apenas \(32\) gerações.

\begin{lstlisting}
A = make_array("                                O ")
B = make_array("                                  ")
i = 0
while i < 32:
    print(make_str(A))
    B = generation(A, B)

    C = A
    A = B
    B = C

    i += 1
\end{lstlisting}

Podemos usar \texttt{CPython} para rodar esse arquivo,
e isso prova que Python é TC. Deixaremos como exercício
para o leitor a implementação de Python em \texttt{Rule 110}.