\section{Parsing}

O parser também tem seu estado interno, e depende
do lexer para receber os lexemas. A técnica de parsing
que usamos se chama parser por descida recursiva
(\textit{Recursive Descent Parser}), e se encaixa
perfeitamente com a gramática que utilizamos.
\index{\textit{Recursive Descent Parser}}
\index{parser}

\begin{lstlisting}
class _Parser:
    def __init__(self, lexer):
        self.lexer = lexer
        self.indent = 0 # numero de espacos
        self.is_tracking = False
        lexer.next() # precisamos popular lexer.word
\end{lstlisting}

Observe, por exemplo, o procedimento que implementa
a produção \verb|Return|:

\begin{lstlisting}
# Return = 'return' Expr.
def _return(parser):
    parser.track("_return")
    res = parser.expect(lexkind.RETURN, "'return' keyword")
    if res.failed():
        return res

    res = parser.expect_prod(_expr, "expression")
    if res.failed():
        return res
    expr = res.value

    n = Node(None, nodekind.RETURN)
    n.leaves = [expr]
    return Result(n, None)
\end{lstlisting}

É uma tradução quase direta da gramática:
primeiramente esperamos a palavra-chave \verb|return|
(linhas 4-6) e
depois esperamos uma expressão \verb|Expr|
(linhas 8-10). O resultado é colocado
em um nó do tipo \verb|nodekind.RETURN|, e é retornado
dentro de um \verb|Result|.

O método \verb|expect| é simples, mas facilita muito a
implementação de um parser, replicamos este a seguir:

\begin{lstlisting}
def expect(self, kind, str):
    if self.word_is(kind):
        return self.consume()
    err = self.error("expected " + str)
    return Result(None, err)
\end{lstlisting}

Além disso, temos alguns utilitários que implementam padrões
comuns na gramática da linguagem, por exemplo,
para implementar listas de qualquer outra produção,
utilizamos:
\index{listas multi-linhas}

\begin{lstlisting}
# implementa:
#    Production {[CommaNL] Production} [CommaNL].
def repeat_comma_list(self, production):
    res = production(self)
    if res.failed() or res.value == None:
        return res
    list = [res.value]
    while self.word_is(lexkind.COMMA):
        _comma_nl(self)
        res = production(self)
        if res.failed():
            return res
        if res.value == None:
            return Result(list, None)
        list += [res.value]
    return Result(list, None)
\end{lstlisting}

\index{indentação}
Os métodos que são responsáveis pela indentação
são \verb|indent_prod| e \verb|same_indent|, que
implementam, respectivamente, o operador de indentação
e o operador de justificação.
O \verb|indent_prod| aumenta o campo \verb|Parser.indent|
antes de executar o procedimento da produção, e retorna
ao campo para o estado anterior quando a produção
terminar de executar:

\begin{lstlisting}
def indent_prod(self, base_indent, production):
    prev_indent = self.indent
    self.indent = base_indent

    if not (self.curr_indent() > self.indent):
        err = self.error("invalid indentation")
        return Result(None, err)

    res = production(self)
    if res.failed():
        return res
    out = res.value
    self.indent = prev_indent
    return Result(out, None)
\end{lstlisting}

É possível ver o uso desse método em procedimentos
como \verb|_while|, \verb|_if| e etc. Sempre
que o operador de indentação for utilizado na gramática.
\index{operador de indentação}

Utilizamos o procedimento \verb|same_indent| no
procedimento \verb|_block| e \verb|_methods|:

\begin{lstlisting}
def _block(parser):
    parser.track("_block")
    statements = []

    base_indent = parser.curr_indent()
    while parser.same_indent(base_indent):
        ...
\end{lstlisting}

Juntos, esses utilitários permitem que a sintaxe sensível
a indentação de Python seja implementada com facilidade.