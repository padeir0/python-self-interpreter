\section{Evaluation}

A função \verb|evaluate| implementa um interpretador em
árvore, que percorre recursivamente a árvore sintática
e executa o código (por isso, na literatura, esses
são referenciados como \textit{Tree-walking interpreters}).
\index{\textit{Tree-walking interpreters}}
\index{interpretador em árvore}

\begin{lstlisting}
def evaluate(builtins, module_map, entry_name, verbose):
    if not (entry_name in module_map):
        msg = "entry module not in module map"
        return Error("", msg, None)

    node = _Call_Node(None, builtins)
    ctx = _Context(module_map, node, builtins)
    if verbose:
        print("\nevaluate\n")
        ctx.toggle_verbose()
    res = _eval_module(ctx, entry_name)
    return res.error
\end{lstlisting}

\index{funções embutidas}
Os argumentos de \verb|evaluate| são os seguintes:
\verb|builtins| é o escopo de funções embutidas;
\verb|module_map| é um dicionário que contém todos os
arquivos de python em um diretório, é chaveado pelo nome
do arquivo, enquanto o valor é o conteúdo do arquivo;
\verb|entry_name| é o nome do arquivo que será executado
primeiro; por fim, \verb|verbose| é um boleano que serve
apenas para fim de debug.

\index{CPython}
O argumento \verb|builtins| é de suma importância para o
funcionamento do projeto: sem ele, não conseguiríamos passar
as funções do \texttt{CPython} para dentro do interpretador.
Ela permite que possamos aproveitar a infraestrutura do
\texttt{CPython} para implementar os objetos embutidos.

Já o argumento \verb|module_map| é o que nos permite
implementar um sistema de módulos com facilidade, veja
que nosso interpretador não permite acessar módulos
em pastas diferentes, isso é intencional, e fornece
uma implementação muito mais simples.
\index{sistema de módulos}

A estrutura do arquivo \verb|evaluator.py| é separada em duas
grandes partes: os objetos e os procedimentos com prefixo
\verb|_eval_|.

Existe um procedimento com prefixo \verb|_eval_| para a
maioria das constantes no arquivo \verb|nodekind.py|.
Desse jeito, separamos a implementação de cada operação
de Python em partes atômicas, permitindo que essas partes
sejam combinadas de acordo com a semântica da linguagem.

Por exemplo, o procedimento \verb|_eval_block| executa,
para cada nó filho, o procedimento \verb|_eval_sttm|,
que executa cada \textit{statement} do bloco. Isso ocorre
na ordem usual de cima para baixo.
\index{statement}

\begin{lstlisting}
def _eval_block(ctx, node):
    if ctx.verbose:
        print("_eval_block")
    i = 0
    while i < len(node.leaves) and not ctx.is_returning:
        sttm = node.leaves[i]
        err = _eval_sttm(ctx, sttm)
        if err != None:
            return err
        i += 1
    return None
\end{lstlisting}

\noindent Veja também que a execução de cada \textit{statement}
pode ser parada caso um \verb|return| seja executado. Para isso,
utilizamos a propriedade \verb|ctx.is_returning|.

\index{evaluator}
O \verb|_Context| é o objeto mais importante do evaluator,
este implementa a pilha de chamadas (\verb|curr_call_node|),
e contém informações globais do programa em execução
(\verb|source_map|, \verb|builtin_scope|, etc). Este objeto
é passado como o argumento \verb|ctx| para todos os
procedimentos com prefixo \verb|_eval_|.

\begin{lstlisting}
class _Context:
    def __init__(self, source_map, call_node, builtin_scope):
        self.builtin_scope = builtin_scope
        self.source_map = source_map
        self.curr_call_node = call_node
        self.evaluated_mods = {}
        self.is_returning = False
        self.verbose = False
    ...
\end{lstlisting}

\index{pilha de chamada} \index{lista encadeada}
O objeto \verb|_Call_Node| implementa a pilha de chamada
a partir de uma lista encadeada. A propriedade \verb|parent|
sempre aponta para o nó anterior da pilha, e é igual a
\verb|None| no primeiro arquivo a ser executado.

\begin{lstlisting}
class _Call_Node:
    def __init__(self, parent, scope):
        self.curr_scope = scope
        self.return_obj = None
        self.parent = parent
\end{lstlisting}

\noindent Veja que toda função tem um escopo, podendo
variar apenas localmente (no escopo de função),
ou globalmente (no escopo de módulo). Por isso, é necessário
que \verb|curr_scope| esteja presente em cada nó da pilha
de chamada, permitindo que a função capture o ambiente
no qual foi declarada. Nesse sentido, todas as funções
no interpretador são \textit{closures}.
\index{\textit{closures}}

Por fim, a propriedade \verb|return_obj| é populada
pelas funções que foram chamadas dentro desse contexto,
é dessa forma que podemos retornar valores de uma função
para outra, mesmo depois que o nó na pilha de chamada seja
eliminado.

\index{escopo} \index{scope}
O \verb|_Scope| por sua vez, é outra lista encadeada,
onde cada nó aponta para o nó anterior \verb|parent|,
exceto no caso do escopo \verb|builtin|, que aponta para
\verb|None|.
Dessa vez, essa lista carrega um dicionário de símbolos
declarados naquele escopo, bem como uma descrição do tipo
de escopo.

\begin{lstlisting}
class _Scope:
    def __init__(self, parent, kind):
        self.kind = kind
        self.parent = parent
        self.name = ""
        self.dict = {}
\end{lstlisting}

O motivo dessa estrutura fica óbvio quando olhamos para o
método responsável por achar um símbolo a partir do escopo
atual:

\begin{lstlisting}
def retrieve(self, name):
    if name in self.dict:
        return Result(self.dict[name], None)
    elif self.parent != None:
        return self.parent.retrieve(name)
    else:
        return Result(None, True)
\end{lstlisting}

\noindent De maneira simplificada: procuramos o nome no
escopo atual: se o nome existir, retornamos o objeto
imediatamente; caso contrário, se o nome não for achado,
procuramos no escopo parente.
Fazemos isso até chegarmos em um escopo cujo
parente é \verb|None| (o escopo \verb|builtin|), nesse caso,
se o nome não existir nesse escopo, retornamos um
\verb|Result| cuja propriedade \verb|error| é não nula,
apenas para notificar que o símbolo não foi achado.

O objeto em questão, que é retornado pelo procedimento
anterior, é do tipo \verb|_Py_Object|. Essa classe encapsula
os objetos do \texttt{CPython} e permite que o interpretador
entenda os tipos do objeto, sem recorrer ao uso da classe
\verb|type|.

\begin{lstlisting}
class _Py_Object:
    def __init__(self, kind, value, mutable):
        self.kind = kind
        self.value = value
        self.mutable = mutable
\end{lstlisting}

\noindent A propriedade \verb|kind| nos permite saber o tipo
do objeto em questão, enquanto \verb|value| contém o próprio
objeto.

Nem todos os objetos são diretamente "importados" de
\texttt{CPython}, por exemplo, os objetos que representam
funções são da classe \verb|_User_Function| (isto é,
a propriedade \verb|value| do \verb|_Py_Object| contém
um objeto do tipo \verb|_User_Function|). Esse objeto
permite que as funções sejam executadas utilizando a
árvore sintática.

\begin{lstlisting}
class _User_Function:
    def __init__(self, name, formal_args, block, parent_scope):
        self.name = name
        self.block = block
        self.formal_args = formal_args
        self.parent_scope = parent_scope
\end{lstlisting}

\noindent Das propriedades, temos:
\verb|formal_args| que é uma lista de identificadores
para os argumentos; \verb|block| que contém a árvore  
sintática do corpo da função; \verb|parent_scope| que é  
o escopo capturado no momento em que a função foi declarada;
e por fim \verb|name|, que é o nome da função.

\index{interpretador em árvore}
Todos esses objetos, e outros de menor importância,
são utilizados dentro dos procedimentos com prefixo
\verb|_eval_| para implementar o interpretador em árvore.