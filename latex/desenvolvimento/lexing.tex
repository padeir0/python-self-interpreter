\section{Lexing}

Com a finalidade de gerar lexemas a partir dos 
caracteres do código fonte, criamos uma máquina de estado
chamada \textit{lexer}.
\index{lexer} \index{máquina de estado}

\begin{lstlisting}
class Lexer:
    def __init__(self, modname, string):
        self.string = string
        self.start = 0
        self.end = 0
        self.range = Range(Position(0, 0), Position(0, 0))
        self.word = None
        self.peeked = None
        self.modname = modname
\end{lstlisting}

O método principal do lexer é o \verb|Lexer.next|. Esse método,
replicado logo em seguida, é responsável por consumir um lexema
do código fonte e coloca-lo em \verb|Lexer.word|.

\begin{lstlisting}
def next(self):
    if self.peeked != None:
        self.word = self.peeked
        self.peeked = None
    else:
        self.word = self._any()
    self._advance()
    return self.word
\end{lstlisting}

Veja que, na linha 6, o retorno do método \verb|self._any|
é atribuído a \verb|self.word|. É nesse método que se encontra
a máquina de estado, implementada como uma longa sequência
de \verb|if|s.

\begin{lstlisting}
def _any(self):
    self._ignore_whitespace()
    r = self._peek_rune()
    if _is_digit(r):
        return self._number()
    elif _is_ident_start(r):
        return self._identifier()
    elif r == "\"":
    ...
\end{lstlisting}

\noindent Não reproduzimos todo o código aqui por ser longo
demais, mas sua estrutura não é complexa: é apenas uma máquina
de estado.

Veja que \verb|Lexer.start| e \verb|Lexer.end| marcam,
respectivamente, o começo e o final de cada lexema. Usamos esses
dois no método \verb|_emit|:
\index{lexema}

\begin{lstlisting}
def _emit(self, kind):
    s = self.string[self.start:self.end]
    return Lexeme(s, kind, self.range.copy())
\end{lstlisting}

\noindent Além disso, para avançar no input,
resetamos esses campos no método \verb|_advance|.

\begin{lstlisting}
def _advance(self):
    self.start = self.end
    self.range.start = self.range.end.copy()
\end{lstlisting}

\index{on-demand-scanner}
Desse jeito, o lexer funciona de acordo com a demanda do
parser, isto é: ele só consome um lexema se o parser pedir.