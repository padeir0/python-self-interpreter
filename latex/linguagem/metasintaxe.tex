\section{Meta-Sintaxe}

\index{\textit{Wirth Syntax Notation}}
Vamos usar a \textit{Wirth Syntax Notation}
definida em \cite{wirth1977WSN}, com extensões
modernas, para descrever o subconjunto de Python que
utilizaremos. Aqui, definiremos apenas as modificações,
e é sugerido ao leitor que leia o artigo
original do Wirth previamente.

A modificação mais simples é que ao invés de usarmos
aspas duplas para descrever terminais, utilizamos aspas
simples, isto é, usamos \verb|'| no lugar de \verb|"|.
Como no exemplo a seguir:

\begin{lstlisting}
Return = 'return' Expr.
\end{lstlisting}

\index{expressões regulares}
Uma segunda modificação conveniente nos permite
escrever expressões regulares, desde que essas estejam
envolvidas por barras \verb|/|. Por exemplo, os
identificadores são descritos da mesma forma que
na linguagem \texttt{C}, e podemos escrever somente:
\index{\texttt{C}}

\begin{lstlisting}
id = /[a-zA-Z_][a-zA-Z0-9_]*/.
\end{lstlisting}

\noindent A sintaxe e semântica
das expressões regulares é a mesma definida
pela biblioteca \texttt{PCRE} \cite{pcre_syntax}
\index{\texttt{PCRE}}

\index{micro-sintaxe}
Por convenção, escrevemos produções de micro-sintaxe
com a primeira letra minúscula. Veremos mais para frente
que essas acabam sendo reconhecidas pelo \textit{Lexer}.
\index{\textit{lexer}}

\index{indentação}
A extensão mais importante é a que nos permite lidar com
a sintaxe sensível a indentação de Python.
Essa é inspirada pelo trabalho dos programadores 
da biblioteca \texttt{Parsec} \cite{adams2014indentation}.
Essa modificação é complexa, mas poderosa, então
acredito ser necessária uma definição própria.
\index{\texttt{Parsec}}

\begin{Def}
\index{nível de indentação}
Definimos o \textbf{nível de indentação de uma produção}
como um número natural
representando a coluna do primeiro caractere significativo
a ser aceito por essa produção.
\end{Def}

No exemplo anterior, o nível de indentação de \verb|Return|
é a coluna do caractere \verb|r| presente na palavra-chave.
Note que o nível de indentação não é fixo,
cada vez que \verb|Return| aceitar uma string do código fonte,
a produção pode adotar um novo valor de indentação,
mas tal valor vai sempre ser herdado do caractere \verb|r|.

Precisamos ainda definir dois operadores. Esses operadores
são usados em conjunto para descrever produções como:

\begin{lstlisting}
While = 'while' Expr ':' NL >Block.
\end{lstlisting}

O leitor deve estar familiarizado
com a sintaxe de Python, e deve
já ter uma intuição sobre o que o operador \verb|>| faz.
Ele serve justamente para dizermos "esse bloco tem que ser
indentado". Definimos isso da seguinte forma:

\begin{Def}
\index{operador de indentação}
Se \(A\) é uma produção definida a partir de \(B\),
de forma que \(B\) é 
precedida pelo \textbf{operador de indentação} \verb|>|, 
então \(B\) deve possuir
um nível de indentação \emph{estritamente}
maior que o nível de indetação de \(A\).
\end{Def}

No caso da produção \verb|While|, o \verb|Bloco|
deve ser indentado
com respeito à palavra-chave \verb|'while'|, isto é,
deve ter nível de indentação estritamente maior
que \verb|While|.

Por fim, para definir \verb|Block|, precisamos do operador
\verb|:|.

\begin{lstlisting}
Block = { :Statement NL }.
\end{lstlisting}

\noindent Essa produção captura a ideia intuitiva que temos
sobre a sintaxe de Python: as instruções de um bloco
devem estar no mesmo nível de indentação.

\begin{Def}
\index{operador de justificação}
Se \(A\) é uma produção definida a partir de \(B\),
de forma que \(B\) é 
precedida pelo \textbf{operador de justificação} \verb|:|, 
então \(B\) deve possuir
um nível de indentação \emph{exatamente igual}
ao nível de indentação de \(A\).
\end{Def}

Note que no caso de \verb|Block|, o nível de indentação
do bloco é o nível do primeiro \verb|Statement|.

Veja que não é necessário \emph{exatamente} 4 espaços para
indentar código em Python \cite{indentation}. Só é necessário
que os \textit{statements}
em um bloco sejam corretamente justificados, e que, quando
ditado pela gramática, o primeiro \textit{statement}
do bloco tenha uma
indentação estritamente maior.
Isso quer dizer que nossa gramática permite
que o código use apenas um espaço como indentação.