\section{Semântica}

Apesar de ser um subconjunto de Python, algumas coisas
são restritas e implementadas de forma diferente. Para
não haver confusões, descrevemos elas aqui.

Para implementar os \verb|for| loops em Python, é necessário o
uso dos métodos \verb|__iter__| e \verb|__next__|, e o segundo
se comunica por meio de exceções. Por consequência, para o
interpretador usar os métodos \verb|__next__| do próprio
CPython, ele precisa entender exceções \cite{iterator_next}, e
elas não serão implementadas.

Não temos tuplas em Spy pois são redundantes: usamos listas
em seu lugar.

Para tratar os erros no parser, existem várias opções.
Idealmente, em Python, usaríamos \textit{Exceptions}, o que
limparia muito a implementação, entretanto, não
vamos implementar \textit{Exceptions}, portanto,
não podemos as usar.

Ao invés disso, tomamos a estratégia de Rust:

\begin{lstlisting}[language=Python]
def _while(parser):
    res = parser.expect(lexkind.WHILE, "while keyword")
    if res.failed():
        return res
\end{lstlisting}

O lado esquerdo de uma atribuição deve conter apenas expressões atribuíveis.
Nesse sentido, essa validação só pode ocorrer em tempo de \textit{runtime}.
Para isso, uma função olha a expressão do lado esquerdo e
tentar achar o objeto, e ela mesma retorna esse erro, caso
o objeto não seja atribuível.

A definição de um \emph{objeto atribuível} em
no subconjunto se dá pelo seguinte:
\begin{itemize}
    \item Se \verb|<e>| é um objeto mutável,
    então \verb|<e>| é atribuível (\verb|a = 1|).
    \item Sendo \verb|<e>| uma expressão que retorna um
    objeto mutável:
    \begin{itemize}
        \item Uma expressão composta com indexação (\verb|<e>[1] = 1|) é atribuível.
        \item Uma expressão composta com acesso a uma propriedade mutável \verb|<e>.prop = 1| é atribuível.
    \end{itemize}
\end{itemize}

Uma propriedade é \emph{mutável} se não é o nome de algum
método. Todos os símbolos importados de outros módulos
são imutáveis.