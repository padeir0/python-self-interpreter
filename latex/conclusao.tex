\chapter{Conclusão}

Apesar de carecer de aplicações práticas, simular
Python dentro de Python nos permite explorar a rica
teoria sobre interpretadores e máquinas abstratas, e,
apesar de ser óbvio que Python era capaz de computação
universal (e, portanto, capaz de se simular), a implementação,
bem como esse documento, servem como aparelho didático para
introdução da teoria de linguagens de programação.

O interpretador está longe de ser perfeito, e existem várias
melhorias que podem ser feitas. Por exemplo, é possível
utilizar a classe \verb|type| e o operador \verb|is|,
para remover a necessidade da classe \verb|_Py_Object|;
além disso, a implementação de exceções não é tão difícil,
e pode melhorar o tratamento de erros no interpretador,
especialmente no parser; por fim, nada impede a implementação
de outras construções linguísticas,
como o \verb|switch| de \texttt{C},
o \verb|export| de \texttt{Modula 2},
o \verb!|>! de \texttt{OCaml},
etc.