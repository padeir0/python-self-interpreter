\documentclass[11pt]{book}
\usepackage[
    a5paper,
    portrait,
    tmargin=2.2cm,
    bmargin=2.5cm,
    headheight=15pt,
    inner=1.5cm, outer=2cm % a encadernação que eu faço tem margens bem comportadas
]{geometry}

\usepackage{graphicx}
\usepackage{titlesec}
\usepackage[brazilian]{babel}
\usepackage{amsthm}
\usepackage{amssymb}
\usepackage{amsmath}
\usepackage{imakeidx}
\usepackage{fancyhdr}
\usepackage[utf8]{inputenc}
\usepackage{csquotes}
\usepackage[hidelinks]{hyperref}
\usepackage{cleveref}
\usepackage[section]{placeins}

\usepackage{listings}
\usepackage{xcolor}

\lstset{
  language=Python,
  basicstyle=\ttfamily\scriptsize, % Fonte pequena e monoespaçada
  keywordstyle=\bfseries,% Palavras-chave em negrito
  showstringspaces=false,% Não mostrar espaços em strings
  numbers=left,          % Números de linha à esquerda
  numberstyle=\tiny,     % Fonte dos números de linha
  tabsize=4,             % Tamanho do tab
  breaklines=true,       % Quebrar linhas longas automaticamente
}

\usepackage[
style=alphabetic,
sorting=ynt,
backend=biber]{biblatex}
\addbibresource{biblio.bib}

\pagestyle{fancy}

\makeindex[title=Índice, columns=2, columnsep=0pt, program=xindy]

\renewcommand{\lstlistingname}{}
\renewcommand{\thechapter}{\arabic{chapter}}
\renewcommand{\thesection}{\arabic{section}}
\renewcommand{\cleardoublepage}{\clearpage} % may god have mercy for this sin
\renewcommand{\chaptermark}[1]{\markboth{\thechapter.\ #1}{}}
\renewcommand{\sectionmark}[1]{\markright{\thesection.\ #1}}
\renewcommand{\headrule}{
\vspace{-9pt}
\hrulefill
\raisebox{0pt}
{\quad\(\therefore\)\quad}%
\hrulefill}
\renewcommand{\thefigure}{\arabic{figure}}
\renewcommand{\thetable}{\arabic{table}}

\renewcommand{\contentsname}{Sumário}
\renewcommand*{\proofname}{Dem}
\theoremstyle{definition}
\newtheorem{Teo}{Teorema}
\newtheorem{Def}{Definição}
\renewcommand{\theDef}{\Roman{Def}}
\renewcommand{\theTeo}{\Roman{Teo}}

\crefname{Def}{}{}
\Crefname{Def}{}{}
\crefname{Teo}{}{}
\Crefname{Teo}{}{}


\makeatletter
  \@removefromreset{section}{chapter}
  \@removefromreset{figure}{chapter}
  \@removefromreset{table}{chapter}
\makeatother



\titleformat{\chapter}[display]
{\normalfont\Large\filcenter\sffamily}
{\titlerule[2pt]\vspace{1.5pt}\titlerule[1pt]\vspace{1pc}\LARGE\thechapter}
{1pc}
{\titlerule\vspace{1pc}\Large}

\titleformat{\section}[block]
{\normalfont \filcenter \bfseries}
{\thesection .}{1pc}{}

\titlespacing{\chapter}{0pc}{-50pt}{40pt}
\titlespacing{\section}{0pc}{*4}{*1.5}

\title{\textbf{Auto-interpretador de Python}}
\author{Artur Iure Vianna Fernandes\\
\small (Matemática)}
\date{\today}

\begin{document}
\emergencystretch 3em
\fancyhead[LE,RO]{\thepage}
\fancyhead[RE]{\rightmark}
\fancyhead[LO]{\leftmark}
\fancyfoot[C]{}

\maketitle

\section*{Resumo}

O objetivo desse trabalho é explorar a ideia de computação
universal,
primeiramente implementando \texttt{Rule 110} em Python,
e em seguida, simulando um subconjunto de Python
usando esse mesmo subconjunto.
\tableofcontents

\chapter{Subconjunto de Python}

\index{interpretador}
Antes de explicar o interpretador, precisamos descrever
muito bem a linguagem que vamos interpretar. Felizmente,
a maior parte do trabalho já foi feito pela \textit{Python
Software Foundation}, mas ainda é necessário explicar algumas
nuanças.
\index{\textit{Python Software Foundation}}

\section{Meta-Sintaxe}

\index{\textit{Wirth Syntax Notation}}
Vamos usar a \textit{Wirth Syntax Notation}
definida em \cite{wirth1977WSN}, com extensões
modernas, para descrever o subconjunto de Python que
utilizaremos. Aqui, definiremos apenas as modificações,
e é sugerido ao leitor que leia o artigo
original do Wirth previamente.

A modificação mais simples é que ao invés de usarmos
aspas duplas para descrever terminais, utilizamos aspas
simples, isto é, usamos \verb|'| no lugar de \verb|"|.
Como no exemplo a seguir:

\begin{lstlisting}
Return = 'return' Expr.
\end{lstlisting}

\index{expressões regulares}
Uma segunda modificação conveniente nos permite
escrever expressões regulares, desde que essas estejam
envolvidas por barras \verb|/|. Por exemplo, os
identificadores são descritos da mesma forma que
na linguagem \texttt{C}, e podemos escrever somente:
\index{\texttt{C}}

\begin{lstlisting}
id = /[a-zA-Z_][a-zA-Z0-9_]*/.
\end{lstlisting}

\noindent A sintaxe e semântica
das expressões regulares é a mesma definida
pela biblioteca \texttt{PCRE} \cite{pcre_syntax}
\index{\texttt{PCRE}}

\index{micro-sintaxe}
Por convenção, escrevemos produções de micro-sintaxe
com a primeira letra minúscula. Veremos mais para frente
que essas acabam sendo reconhecidas pelo \textit{Lexer}.
\index{\textit{lexer}}

\index{indentação}
A extensão mais importante é a que nos permite lidar com
a sintaxe sensível a indentação de Python.
Essa é inspirada pelo trabalho dos programadores 
da biblioteca \texttt{Parsec} \cite{adams2014indentation}.
Essa modificação é complexa, mas poderosa, então
acredito ser necessária uma definição própria.
\index{\texttt{Parsec}}

\begin{Def}
\index{nível de indentação}
Definimos o \textbf{nível de indentação de uma produção}
como um número natural
representando a coluna do primeiro caractere significativo
a ser aceito por essa produção.
\end{Def}

No exemplo anterior, o nível de indentação de \verb|Return|
é a coluna do caractere \verb|r| presente na palavra-chave.
Note que o nível de indentação não é fixo,
cada vez que \verb|Return| aceitar uma string do código fonte,
a produção pode adotar um novo valor de indentação,
mas tal valor vai sempre ser herdado do caractere \verb|r|.

Precisamos ainda definir dois operadores. Esses operadores
são usados em conjunto para descrever produções como:

\begin{lstlisting}
While = 'while' Expr ':' NL >Block.
\end{lstlisting}

O leitor deve estar familiarizado
com a sintaxe de Python, e deve
já ter uma intuição sobre o que o operador \verb|>| faz.
Ele serve justamente para dizermos "esse bloco tem que ser
indentado". Definimos isso da seguinte forma:

\begin{Def}
\index{operador de indentação}
Se \(A\) é uma produção definida a partir de \(B\),
de forma que \(B\) é 
precedida pelo \textbf{operador de indentação} \verb|>|, 
então \(B\) deve possuir
um nível de indentação \emph{estritamente}
maior que o nível de indetação de \(A\).
\end{Def}

No caso da produção \verb|While|, o \verb|Bloco|
deve ser indentado
com respeito à palavra-chave \verb|'while'|, isto é,
deve ter nível de indentação estritamente maior
que \verb|While|.

Por fim, para definir \verb|Block|, precisamos do operador
\verb|:|.

\begin{lstlisting}
Block = { :Statement NL }.
\end{lstlisting}

\noindent Essa produção captura a ideia intuitiva que temos
sobre a sintaxe de Python: as instruções de um bloco
devem estar no mesmo nível de indentação.

\begin{Def}
\index{operador de justificação}
Se \(A\) é uma produção definida a partir de \(B\),
de forma que \(B\) é 
precedida pelo \textbf{operador de justificação} \verb|:|, 
então \(B\) deve possuir
um nível de indentação \emph{exatamente igual}
ao nível de indentação de \(A\).
\end{Def}

Note que no caso de \verb|Block|, o nível de indentação
do bloco é o nível do primeiro \verb|Statement|.

Veja que não é necessário \emph{exatamente} 4 espaços para
indentar código em Python \cite{indentation}. Só é necessário
que os \textit{statements}
em um bloco sejam corretamente justificados, e que, quando
ditado pela gramática, o primeiro \textit{statement}
do bloco tenha uma
indentação estritamente maior.
Isso quer dizer que nossa gramática permite
que o código use apenas um espaço como indentação.
\section{Gramática}

Antes de definirmos as estruturas gramaticais,
precisamos primeiramente definir quais caracteres
podem ser ignorados. Para isso, definimos a produção
\verb|WhiteSpace|, que inclui espaços, \textit{carriage returns}
e comentários.
\index{whitespace}

\begin{lstlisting}
WhiteSpace = ' ' | '\r' | Comment.
Comment = '#' {ascii} nl.
nl = '\n'.
\end{lstlisting}

\index{statement}
Cada módulo em Python é simplesmente um bloco de código,
e blocos são sequências de \textit{statements} justificados.
A tradução de \textit{statement} para português é difícil,
mas é melhor aproximada pela palavra \textit{instrução}.

\begin{lstlisting}
Module = Block.
Block = { :Statement NL }.
NL = nl {nl}.
\end{lstlisting}

\index{nova linha} \index{line break}
\noindent Veja que definimos \verb|NL|
(leia como: \textbf{N}ova \textbf{L}inha)
permitindo que em qualquer lugar que ocorra uma nova linha,
possam ocorrer outras em sequência. Em \texttt{CPython},
linhas em branco não emitem tokens de nova-linha
\cite{blank_lines}, isso é uma heurística que não precisamos
adotar.

A definição de \verb|Statement| é complexa e é
quebrada em várias subproduções.

\begin{lstlisting}
Statement = While  | DoWhile | If  | Atrib_Expr
          | Return | Class   | Func
          | Import | FromImport | Pass.

Pass = 'pass'.

Import = 'import' IdList.
FromImport = 'from' id 'import' IdList.
IdList = id {CommaNL id} [CommaNL].

DoWhile = 'do'':' NL >Block 'while' Expr.
While = 'while' Expr ':' NL >Block.

If = 'if' Expr ':' NL >Block {:Elif} [:Else].
Elif = 'elif' Expr ':' NL >Block.
Else = 'else' ':' NL >Block.

Atrib_Expr = Expr [assign_Op Expr].
assign_Op = '=' | '+=' | '-=' | '*=' | '/=' | '%='.

Return = 'return' Expr.

Class = 'class' id ':' NL >Methods.
Methods = {:Func}.

Func = 'def' id Arguments ':' NL >Block.
Arguments = '(' [NL] [ArgList] ')'.
ArgList = Arg {CommaNL Arg} [CommaNL].
Arg = 'self' | id.
\end{lstlisting}

Uma estratégia usada acima, que também é usada na
produção a seguir, permite que a sintaxe aceite
listas em múltiplas linhas.
Para permitir essas listas,
\texttt{CPython} faz com que o lexer pare de emitir tokens
de nova-linha dentro de delimitadores
\cite{implicit_line_joining}, isso é outra heurística
que não precisamos adotar.
\index{\texttt{CPython}}
\index{implicit line joining}

\begin{lstlisting}
ExprList = Expr {CommaNL Expr} [CommaNL].
CommaNL = ',' [NL].
\end{lstlisting}

\index{Pascal}
Por fim, finalmente definimos a sintaxe de uma expressão.
Note que a precedência dos operadores está diretamente
embutida na gramática. Essa estratégia foi disseminada
por Wirth nas gramáticas de \texttt{Pascal}, e permite uma
quantidade limitada de níveis de precedência sem a necessidade
de usar tabelas de precedência.

\begin{lstlisting}
Expr = And {'or' And}.
And = Comp {'and' Comp}.
Comp = Sum {compOp Sum}.
compOp = '==' | '!=' | '>' | '>=' | '<' | '<=' | 'in'.
Sum = Mult {sumOp Mult}.
sumOp = '+' | '-'.
Mult = UnaryPrefix {multOp UnaryPrefix}.
multOp = '*' | '/' | '%'.
UnaryPrefix = {Prefix} UnarySuffix.
UnarySuffix = Term {Suffix}.
Term = 'self' | 'None' | bool | num
       | str  | id     | NestedExpr
       | Dict | List.
NestedExpr = '(' Expr ')'.
\end{lstlisting}

Descrevemos indexação e chamadas de função
como operadores unários de sufixo. 

\begin{lstlisting}
prefix = 'not' | '-'.
Suffix = Call
       | DotAccess
       | Index.
Call = '(' [NL] [ExprList] ')'.
Index = '[' [NL] Expr [':' Expr] ']'.
DotAccess = '.' id.
List = '[' [NL] ExprList ']'.
Dict = '{' [NL] KeyValue_List '}'.

KeyValue_List = KeyValue_Expr {CommaNL KeyValue_Expr} [CommaNL].
KeyValue_Expr = Expr [':' Expr].
\end{lstlisting}

\index{literais}
Por fim, definimos os literais, abusando de expressões
regulares.

\begin{lstlisting}
bool = 'True' | 'False'.
num = /[0-9]+/.
str = '"' insideString* '"'.
insideString = ascii | escapes.
escapes = '\\n' | '\\"'
id = /[a-zA-Z_][a-zA-Z0-9_]*/.
\end{lstlisting}

Veja que, por causa das complexidades de verificar as regras de
indentação quando tabs estão envolvidas \cite{tab_error},
essas são completamente proibidas no código fonte.
\index{tabs}
\section{Semântica}

Apesar de ser um subconjunto de Python, algumas coisas
são restritas e implementadas de forma diferente. Para
não haver confusões, descrevemos elas aqui.

Para implementar os \verb|for| loops em Python, é necessário o
uso dos métodos \verb|__iter__| e \verb|__next__|, e o segundo
se comunica por meio de exceções. Por consequência, para o
interpretador usar os métodos \verb|__next__| do próprio
CPython, ele precisa entender exceções \cite{iterator_next}, e
elas não serão implementadas.

Não temos tuplas em Spy pois são redundantes: usamos listas
em seu lugar.

Para tratar os erros no parser, existem várias opções.
Idealmente, em Python, usaríamos \textit{Exceptions}, o que
limparia muito a implementação, entretanto, não
vamos implementar \textit{Exceptions}, portanto,
não podemos as usar.

Ao invés disso, tomamos a estratégia de Rust:

\begin{lstlisting}[language=Python]
def _while(parser):
    res = parser.expect(lexkind.WHILE, "while keyword")
    if res.failed():
        return res
\end{lstlisting}

O lado esquerdo de uma atribuição deve conter apenas expressões atribuíveis.
Nesse sentido, essa validação só pode ocorrer em tempo de \textit{runtime}.
Para isso, uma função olha a expressão do lado esquerdo e
tentar achar o objeto, e ela mesma retorna esse erro, caso
o objeto não seja atribuível.

A definição de um \emph{objeto atribuível} em
no subconjunto se dá pelo seguinte:
\begin{itemize}
    \item Se \verb|<e>| é um objeto mutável,
    então \verb|<e>| é atribuível (\verb|a = 1|).
    \item Sendo \verb|<e>| uma expressão que retorna um
    objeto mutável:
    \begin{itemize}
        \item Uma expressão composta com indexação (\verb|<e>[1] = 1|) é atribuível.
        \item Uma expressão composta com acesso a uma propriedade mutável \verb|<e>.prop = 1| é atribuível.
    \end{itemize}
\end{itemize}

Uma propriedade é \emph{mutável} se não é o nome de algum
método. Todos os símbolos importados de outros módulos
são imutáveis.
\chapter{Subconjunto de Python}

\index{interpretador}
Antes de explicar o interpretador, precisamos descrever
muito bem a linguagem que vamos interpretar. Felizmente,
a maior parte do trabalho já foi feito pela \textit{Python
Software Foundation}, mas ainda é necessário explicar algumas
nuanças.
\index{\textit{Python Software Foundation}}

\section{Meta-Sintaxe}

\index{\textit{Wirth Syntax Notation}}
Vamos usar a \textit{Wirth Syntax Notation}
definida em \cite{wirth1977WSN}, com extensões
modernas, para descrever o subconjunto de Python que
utilizaremos. Aqui, definiremos apenas as modificações,
e é sugerido ao leitor que leia o artigo
original do Wirth previamente.

A modificação mais simples é que ao invés de usarmos
aspas duplas para descrever terminais, utilizamos aspas
simples, isto é, usamos \verb|'| no lugar de \verb|"|.
Como no exemplo a seguir:

\begin{lstlisting}
Return = 'return' Expr.
\end{lstlisting}

\index{expressões regulares}
Uma segunda modificação conveniente nos permite
escrever expressões regulares, desde que essas estejam
envolvidas por barras \verb|/|. Por exemplo, os
identificadores são descritos da mesma forma que
na linguagem \texttt{C}, e podemos escrever somente:
\index{\texttt{C}}

\begin{lstlisting}
id = /[a-zA-Z_][a-zA-Z0-9_]*/.
\end{lstlisting}

\noindent A sintaxe e semântica
das expressões regulares é a mesma definida
pela biblioteca \texttt{PCRE} \cite{pcre_syntax}
\index{\texttt{PCRE}}

\index{micro-sintaxe}
Por convenção, escrevemos produções de micro-sintaxe
com a primeira letra minúscula. Veremos mais para frente
que essas acabam sendo reconhecidas pelo \textit{Lexer}.
\index{\textit{lexer}}

\index{indentação}
A extensão mais importante é a que nos permite lidar com
a sintaxe sensível a indentação de Python.
Essa é inspirada pelo trabalho dos programadores 
da biblioteca \texttt{Parsec} \cite{adams2014indentation}.
Essa modificação é complexa, mas poderosa, então
acredito ser necessária uma definição própria.
\index{\texttt{Parsec}}

\begin{Def}
\index{nível de indentação}
Definimos o \textbf{nível de indentação de uma produção}
como um número natural
representando a coluna do primeiro caractere significativo
a ser aceito por essa produção.
\end{Def}

No exemplo anterior, o nível de indentação de \verb|Return|
é a coluna do caractere \verb|r| presente na palavra-chave.
Note que o nível de indentação não é fixo,
cada vez que \verb|Return| aceitar uma string do código fonte,
a produção pode adotar um novo valor de indentação,
mas tal valor vai sempre ser herdado do caractere \verb|r|.

Precisamos ainda definir dois operadores. Esses operadores
são usados em conjunto para descrever produções como:

\begin{lstlisting}
While = 'while' Expr ':' NL >Block.
\end{lstlisting}

O leitor deve estar familiarizado
com a sintaxe de Python, e deve
já ter uma intuição sobre o que o operador \verb|>| faz.
Ele serve justamente para dizermos "esse bloco tem que ser
indentado". Definimos isso da seguinte forma:

\begin{Def}
\index{operador de indentação}
Se \(A\) é uma produção definida a partir de \(B\),
de forma que \(B\) é 
precedida pelo \textbf{operador de indentação} \verb|>|, 
então \(B\) deve possuir
um nível de indentação \emph{estritamente}
maior que o nível de indetação de \(A\).
\end{Def}

No caso da produção \verb|While|, o \verb|Bloco|
deve ser indentado
com respeito à palavra-chave \verb|'while'|, isto é,
deve ter nível de indentação estritamente maior
que \verb|While|.

Por fim, para definir \verb|Block|, precisamos do operador
\verb|:|.

\begin{lstlisting}
Block = { :Statement NL }.
\end{lstlisting}

\noindent Essa produção captura a ideia intuitiva que temos
sobre a sintaxe de Python: as instruções de um bloco
devem estar no mesmo nível de indentação.

\begin{Def}
\index{operador de justificação}
Se \(A\) é uma produção definida a partir de \(B\),
de forma que \(B\) é 
precedida pelo \textbf{operador de justificação} \verb|:|, 
então \(B\) deve possuir
um nível de indentação \emph{exatamente igual}
ao nível de indentação de \(A\).
\end{Def}

Note que no caso de \verb|Block|, o nível de indentação
do bloco é o nível do primeiro \verb|Statement|.

Veja que não é necessário \emph{exatamente} 4 espaços para
indentar código em Python \cite{indentation}. Só é necessário
que os \textit{statements}
em um bloco sejam corretamente justificados, e que, quando
ditado pela gramática, o primeiro \textit{statement}
do bloco tenha uma
indentação estritamente maior.
Isso quer dizer que nossa gramática permite
que o código use apenas um espaço como indentação.
\section{Gramática}

Antes de definirmos as estruturas gramaticais,
precisamos primeiramente definir quais caracteres
podem ser ignorados. Para isso, definimos a produção
\verb|WhiteSpace|, que inclui espaços, \textit{carriage returns}
e comentários.
\index{whitespace}

\begin{lstlisting}
WhiteSpace = ' ' | '\r' | Comment.
Comment = '#' {ascii} nl.
nl = '\n'.
\end{lstlisting}

\index{statement}
Cada módulo em Python é simplesmente um bloco de código,
e blocos são sequências de \textit{statements} justificados.
A tradução de \textit{statement} para português é difícil,
mas é melhor aproximada pela palavra \textit{instrução}.

\begin{lstlisting}
Module = Block.
Block = { :Statement NL }.
NL = nl {nl}.
\end{lstlisting}

\index{nova linha} \index{line break}
\noindent Veja que definimos \verb|NL|
(leia como: \textbf{N}ova \textbf{L}inha)
permitindo que em qualquer lugar que ocorra uma nova linha,
possam ocorrer outras em sequência. Em \texttt{CPython},
linhas em branco não emitem tokens de nova-linha
\cite{blank_lines}, isso é uma heurística que não precisamos
adotar.

A definição de \verb|Statement| é complexa e é
quebrada em várias subproduções.

\begin{lstlisting}
Statement = While  | DoWhile | If  | Atrib_Expr
          | Return | Class   | Func
          | Import | FromImport | Pass.

Pass = 'pass'.

Import = 'import' IdList.
FromImport = 'from' id 'import' IdList.
IdList = id {CommaNL id} [CommaNL].

DoWhile = 'do'':' NL >Block 'while' Expr.
While = 'while' Expr ':' NL >Block.

If = 'if' Expr ':' NL >Block {:Elif} [:Else].
Elif = 'elif' Expr ':' NL >Block.
Else = 'else' ':' NL >Block.

Atrib_Expr = Expr [assign_Op Expr].
assign_Op = '=' | '+=' | '-=' | '*=' | '/=' | '%='.

Return = 'return' Expr.

Class = 'class' id ':' NL >Methods.
Methods = {:Func}.

Func = 'def' id Arguments ':' NL >Block.
Arguments = '(' [NL] [ArgList] ')'.
ArgList = Arg {CommaNL Arg} [CommaNL].
Arg = 'self' | id.
\end{lstlisting}

Uma estratégia usada acima, que também é usada na
produção a seguir, permite que a sintaxe aceite
listas em múltiplas linhas.
Para permitir essas listas,
\texttt{CPython} faz com que o lexer pare de emitir tokens
de nova-linha dentro de delimitadores
\cite{implicit_line_joining}, isso é outra heurística
que não precisamos adotar.
\index{\texttt{CPython}}
\index{implicit line joining}

\begin{lstlisting}
ExprList = Expr {CommaNL Expr} [CommaNL].
CommaNL = ',' [NL].
\end{lstlisting}

\index{Pascal}
Por fim, finalmente definimos a sintaxe de uma expressão.
Note que a precedência dos operadores está diretamente
embutida na gramática. Essa estratégia foi disseminada
por Wirth nas gramáticas de \texttt{Pascal}, e permite uma
quantidade limitada de níveis de precedência sem a necessidade
de usar tabelas de precedência.

\begin{lstlisting}
Expr = And {'or' And}.
And = Comp {'and' Comp}.
Comp = Sum {compOp Sum}.
compOp = '==' | '!=' | '>' | '>=' | '<' | '<=' | 'in'.
Sum = Mult {sumOp Mult}.
sumOp = '+' | '-'.
Mult = UnaryPrefix {multOp UnaryPrefix}.
multOp = '*' | '/' | '%'.
UnaryPrefix = {Prefix} UnarySuffix.
UnarySuffix = Term {Suffix}.
Term = 'self' | 'None' | bool | num
       | str  | id     | NestedExpr
       | Dict | List.
NestedExpr = '(' Expr ')'.
\end{lstlisting}

Descrevemos indexação e chamadas de função
como operadores unários de sufixo. 

\begin{lstlisting}
prefix = 'not' | '-'.
Suffix = Call
       | DotAccess
       | Index.
Call = '(' [NL] [ExprList] ')'.
Index = '[' [NL] Expr [':' Expr] ']'.
DotAccess = '.' id.
List = '[' [NL] ExprList ']'.
Dict = '{' [NL] KeyValue_List '}'.

KeyValue_List = KeyValue_Expr {CommaNL KeyValue_Expr} [CommaNL].
KeyValue_Expr = Expr [':' Expr].
\end{lstlisting}

\index{literais}
Por fim, definimos os literais, abusando de expressões
regulares.

\begin{lstlisting}
bool = 'True' | 'False'.
num = /[0-9]+/.
str = '"' insideString* '"'.
insideString = ascii | escapes.
escapes = '\\n' | '\\"'
id = /[a-zA-Z_][a-zA-Z0-9_]*/.
\end{lstlisting}

Veja que, por causa das complexidades de verificar as regras de
indentação quando tabs estão envolvidas \cite{tab_error},
essas são completamente proibidas no código fonte.
\index{tabs}
\section{Semântica}

Apesar de ser um subconjunto de Python, algumas coisas
são restritas e implementadas de forma diferente. Para
não haver confusões, descrevemos elas aqui.

Para implementar os \verb|for| loops em Python, é necessário o
uso dos métodos \verb|__iter__| e \verb|__next__|, e o segundo
se comunica por meio de exceções. Por consequência, para o
interpretador usar os métodos \verb|__next__| do próprio
CPython, ele precisa entender exceções \cite{iterator_next}, e
elas não serão implementadas.

Não temos tuplas em Spy pois são redundantes: usamos listas
em seu lugar.

Para tratar os erros no parser, existem várias opções.
Idealmente, em Python, usaríamos \textit{Exceptions}, o que
limparia muito a implementação, entretanto, não
vamos implementar \textit{Exceptions}, portanto,
não podemos as usar.

Ao invés disso, tomamos a estratégia de Rust:

\begin{lstlisting}[language=Python]
def _while(parser):
    res = parser.expect(lexkind.WHILE, "while keyword")
    if res.failed():
        return res
\end{lstlisting}

O lado esquerdo de uma atribuição deve conter apenas expressões atribuíveis.
Nesse sentido, essa validação só pode ocorrer em tempo de \textit{runtime}.
Para isso, uma função olha a expressão do lado esquerdo e
tentar achar o objeto, e ela mesma retorna esse erro, caso
o objeto não seja atribuível.

A definição de um \emph{objeto atribuível} em
no subconjunto se dá pelo seguinte:
\begin{itemize}
    \item Se \verb|<e>| é um objeto mutável,
    então \verb|<e>| é atribuível (\verb|a = 1|).
    \item Sendo \verb|<e>| uma expressão que retorna um
    objeto mutável:
    \begin{itemize}
        \item Uma expressão composta com indexação (\verb|<e>[1] = 1|) é atribuível.
        \item Uma expressão composta com acesso a uma propriedade mutável \verb|<e>.prop = 1| é atribuível.
    \end{itemize}
\end{itemize}

Uma propriedade é \emph{mutável} se não é o nome de algum
método. Todos os símbolos importados de outros módulos
são imutáveis.
\chapter{Subconjunto de Python}

\index{interpretador}
Antes de explicar o interpretador, precisamos descrever
muito bem a linguagem que vamos interpretar. Felizmente,
a maior parte do trabalho já foi feito pela \textit{Python
Software Foundation}, mas ainda é necessário explicar algumas
nuanças.
\index{\textit{Python Software Foundation}}

\section{Meta-Sintaxe}

\index{\textit{Wirth Syntax Notation}}
Vamos usar a \textit{Wirth Syntax Notation}
definida em \cite{wirth1977WSN}, com extensões
modernas, para descrever o subconjunto de Python que
utilizaremos. Aqui, definiremos apenas as modificações,
e é sugerido ao leitor que leia o artigo
original do Wirth previamente.

A modificação mais simples é que ao invés de usarmos
aspas duplas para descrever terminais, utilizamos aspas
simples, isto é, usamos \verb|'| no lugar de \verb|"|.
Como no exemplo a seguir:

\begin{lstlisting}
Return = 'return' Expr.
\end{lstlisting}

\index{expressões regulares}
Uma segunda modificação conveniente nos permite
escrever expressões regulares, desde que essas estejam
envolvidas por barras \verb|/|. Por exemplo, os
identificadores são descritos da mesma forma que
na linguagem \texttt{C}, e podemos escrever somente:
\index{\texttt{C}}

\begin{lstlisting}
id = /[a-zA-Z_][a-zA-Z0-9_]*/.
\end{lstlisting}

\noindent A sintaxe e semântica
das expressões regulares é a mesma definida
pela biblioteca \texttt{PCRE} \cite{pcre_syntax}
\index{\texttt{PCRE}}

\index{micro-sintaxe}
Por convenção, escrevemos produções de micro-sintaxe
com a primeira letra minúscula. Veremos mais para frente
que essas acabam sendo reconhecidas pelo \textit{Lexer}.
\index{\textit{lexer}}

\index{indentação}
A extensão mais importante é a que nos permite lidar com
a sintaxe sensível a indentação de Python.
Essa é inspirada pelo trabalho dos programadores 
da biblioteca \texttt{Parsec} \cite{adams2014indentation}.
Essa modificação é complexa, mas poderosa, então
acredito ser necessária uma definição própria.
\index{\texttt{Parsec}}

\begin{Def}
\index{nível de indentação}
Definimos o \textbf{nível de indentação de uma produção}
como um número natural
representando a coluna do primeiro caractere significativo
a ser aceito por essa produção.
\end{Def}

No exemplo anterior, o nível de indentação de \verb|Return|
é a coluna do caractere \verb|r| presente na palavra-chave.
Note que o nível de indentação não é fixo,
cada vez que \verb|Return| aceitar uma string do código fonte,
a produção pode adotar um novo valor de indentação,
mas tal valor vai sempre ser herdado do caractere \verb|r|.

Precisamos ainda definir dois operadores. Esses operadores
são usados em conjunto para descrever produções como:

\begin{lstlisting}
While = 'while' Expr ':' NL >Block.
\end{lstlisting}

O leitor deve estar familiarizado
com a sintaxe de Python, e deve
já ter uma intuição sobre o que o operador \verb|>| faz.
Ele serve justamente para dizermos "esse bloco tem que ser
indentado". Definimos isso da seguinte forma:

\begin{Def}
\index{operador de indentação}
Se \(A\) é uma produção definida a partir de \(B\),
de forma que \(B\) é 
precedida pelo \textbf{operador de indentação} \verb|>|, 
então \(B\) deve possuir
um nível de indentação \emph{estritamente}
maior que o nível de indetação de \(A\).
\end{Def}

No caso da produção \verb|While|, o \verb|Bloco|
deve ser indentado
com respeito à palavra-chave \verb|'while'|, isto é,
deve ter nível de indentação estritamente maior
que \verb|While|.

Por fim, para definir \verb|Block|, precisamos do operador
\verb|:|.

\begin{lstlisting}
Block = { :Statement NL }.
\end{lstlisting}

\noindent Essa produção captura a ideia intuitiva que temos
sobre a sintaxe de Python: as instruções de um bloco
devem estar no mesmo nível de indentação.

\begin{Def}
\index{operador de justificação}
Se \(A\) é uma produção definida a partir de \(B\),
de forma que \(B\) é 
precedida pelo \textbf{operador de justificação} \verb|:|, 
então \(B\) deve possuir
um nível de indentação \emph{exatamente igual}
ao nível de indentação de \(A\).
\end{Def}

Note que no caso de \verb|Block|, o nível de indentação
do bloco é o nível do primeiro \verb|Statement|.

Veja que não é necessário \emph{exatamente} 4 espaços para
indentar código em Python \cite{indentation}. Só é necessário
que os \textit{statements}
em um bloco sejam corretamente justificados, e que, quando
ditado pela gramática, o primeiro \textit{statement}
do bloco tenha uma
indentação estritamente maior.
Isso quer dizer que nossa gramática permite
que o código use apenas um espaço como indentação.
\section{Gramática}

Antes de definirmos as estruturas gramaticais,
precisamos primeiramente definir quais caracteres
podem ser ignorados. Para isso, definimos a produção
\verb|WhiteSpace|, que inclui espaços, \textit{carriage returns}
e comentários.
\index{whitespace}

\begin{lstlisting}
WhiteSpace = ' ' | '\r' | Comment.
Comment = '#' {ascii} nl.
nl = '\n'.
\end{lstlisting}

\index{statement}
Cada módulo em Python é simplesmente um bloco de código,
e blocos são sequências de \textit{statements} justificados.
A tradução de \textit{statement} para português é difícil,
mas é melhor aproximada pela palavra \textit{instrução}.

\begin{lstlisting}
Module = Block.
Block = { :Statement NL }.
NL = nl {nl}.
\end{lstlisting}

\index{nova linha} \index{line break}
\noindent Veja que definimos \verb|NL|
(leia como: \textbf{N}ova \textbf{L}inha)
permitindo que em qualquer lugar que ocorra uma nova linha,
possam ocorrer outras em sequência. Em \texttt{CPython},
linhas em branco não emitem tokens de nova-linha
\cite{blank_lines}, isso é uma heurística que não precisamos
adotar.

A definição de \verb|Statement| é complexa e é
quebrada em várias subproduções.

\begin{lstlisting}
Statement = While  | DoWhile | If  | Atrib_Expr
          | Return | Class   | Func
          | Import | FromImport | Pass.

Pass = 'pass'.

Import = 'import' IdList.
FromImport = 'from' id 'import' IdList.
IdList = id {CommaNL id} [CommaNL].

DoWhile = 'do'':' NL >Block 'while' Expr.
While = 'while' Expr ':' NL >Block.

If = 'if' Expr ':' NL >Block {:Elif} [:Else].
Elif = 'elif' Expr ':' NL >Block.
Else = 'else' ':' NL >Block.

Atrib_Expr = Expr [assign_Op Expr].
assign_Op = '=' | '+=' | '-=' | '*=' | '/=' | '%='.

Return = 'return' Expr.

Class = 'class' id ':' NL >Methods.
Methods = {:Func}.

Func = 'def' id Arguments ':' NL >Block.
Arguments = '(' [NL] [ArgList] ')'.
ArgList = Arg {CommaNL Arg} [CommaNL].
Arg = 'self' | id.
\end{lstlisting}

Uma estratégia usada acima, que também é usada na
produção a seguir, permite que a sintaxe aceite
listas em múltiplas linhas.
Para permitir essas listas,
\texttt{CPython} faz com que o lexer pare de emitir tokens
de nova-linha dentro de delimitadores
\cite{implicit_line_joining}, isso é outra heurística
que não precisamos adotar.
\index{\texttt{CPython}}
\index{implicit line joining}

\begin{lstlisting}
ExprList = Expr {CommaNL Expr} [CommaNL].
CommaNL = ',' [NL].
\end{lstlisting}

\index{Pascal}
Por fim, finalmente definimos a sintaxe de uma expressão.
Note que a precedência dos operadores está diretamente
embutida na gramática. Essa estratégia foi disseminada
por Wirth nas gramáticas de \texttt{Pascal}, e permite uma
quantidade limitada de níveis de precedência sem a necessidade
de usar tabelas de precedência.

\begin{lstlisting}
Expr = And {'or' And}.
And = Comp {'and' Comp}.
Comp = Sum {compOp Sum}.
compOp = '==' | '!=' | '>' | '>=' | '<' | '<=' | 'in'.
Sum = Mult {sumOp Mult}.
sumOp = '+' | '-'.
Mult = UnaryPrefix {multOp UnaryPrefix}.
multOp = '*' | '/' | '%'.
UnaryPrefix = {Prefix} UnarySuffix.
UnarySuffix = Term {Suffix}.
Term = 'self' | 'None' | bool | num
       | str  | id     | NestedExpr
       | Dict | List.
NestedExpr = '(' Expr ')'.
\end{lstlisting}

Descrevemos indexação e chamadas de função
como operadores unários de sufixo. 

\begin{lstlisting}
prefix = 'not' | '-'.
Suffix = Call
       | DotAccess
       | Index.
Call = '(' [NL] [ExprList] ')'.
Index = '[' [NL] Expr [':' Expr] ']'.
DotAccess = '.' id.
List = '[' [NL] ExprList ']'.
Dict = '{' [NL] KeyValue_List '}'.

KeyValue_List = KeyValue_Expr {CommaNL KeyValue_Expr} [CommaNL].
KeyValue_Expr = Expr [':' Expr].
\end{lstlisting}

\index{literais}
Por fim, definimos os literais, abusando de expressões
regulares.

\begin{lstlisting}
bool = 'True' | 'False'.
num = /[0-9]+/.
str = '"' insideString* '"'.
insideString = ascii | escapes.
escapes = '\\n' | '\\"'
id = /[a-zA-Z_][a-zA-Z0-9_]*/.
\end{lstlisting}

Veja que, por causa das complexidades de verificar as regras de
indentação quando tabs estão envolvidas \cite{tab_error},
essas são completamente proibidas no código fonte.
\index{tabs}
\section{Semântica}

Apesar de ser um subconjunto de Python, algumas coisas
são restritas e implementadas de forma diferente. Para
não haver confusões, descrevemos elas aqui.

Para implementar os \verb|for| loops em Python, é necessário o
uso dos métodos \verb|__iter__| e \verb|__next__|, e o segundo
se comunica por meio de exceções. Por consequência, para o
interpretador usar os métodos \verb|__next__| do próprio
CPython, ele precisa entender exceções \cite{iterator_next}, e
elas não serão implementadas.

Não temos tuplas em Spy pois são redundantes: usamos listas
em seu lugar.

Para tratar os erros no parser, existem várias opções.
Idealmente, em Python, usaríamos \textit{Exceptions}, o que
limparia muito a implementação, entretanto, não
vamos implementar \textit{Exceptions}, portanto,
não podemos as usar.

Ao invés disso, tomamos a estratégia de Rust:

\begin{lstlisting}[language=Python]
def _while(parser):
    res = parser.expect(lexkind.WHILE, "while keyword")
    if res.failed():
        return res
\end{lstlisting}

O lado esquerdo de uma atribuição deve conter apenas expressões atribuíveis.
Nesse sentido, essa validação só pode ocorrer em tempo de \textit{runtime}.
Para isso, uma função olha a expressão do lado esquerdo e
tentar achar o objeto, e ela mesma retorna esse erro, caso
o objeto não seja atribuível.

A definição de um \emph{objeto atribuível} em
no subconjunto se dá pelo seguinte:
\begin{itemize}
    \item Se \verb|<e>| é um objeto mutável,
    então \verb|<e>| é atribuível (\verb|a = 1|).
    \item Sendo \verb|<e>| uma expressão que retorna um
    objeto mutável:
    \begin{itemize}
        \item Uma expressão composta com indexação (\verb|<e>[1] = 1|) é atribuível.
        \item Uma expressão composta com acesso a uma propriedade mutável \verb|<e>.prop = 1| é atribuível.
    \end{itemize}
\end{itemize}

Uma propriedade é \emph{mutável} se não é o nome de algum
método. Todos os símbolos importados de outros módulos
são imutáveis.
\chapter{Conclusão}

Apesar de carecer de aplicações práticas, simular
Python dentro de Python nos permite explorar a rica
teoria sobre interpretadores e máquinas abstratas, e,
apesar de ser óbvio que Python era capaz de computação
universal (e, portanto, capaz de se simular), a implementação,
bem como esse documento, servem como aparelho didático para
introdução da teoria de linguagens de programação.

O interpretador está longe de ser perfeito, e existem várias
melhorias que podem ser feitas. Por exemplo, é possível
utilizar a classe \verb|type| e o operador \verb|is|,
para remover a necessidade da classe \verb|_Py_Object|;
além disso, a implementação de exceções não é tão difícil,
e pode melhorar o tratamento de erros no interpretador,
especialmente no parser; por fim, nada impede a implementação
de outras construções linguísticas,
como o \verb|switch| de \texttt{C},
o \verb|export| de \texttt{Modula 2},
o \verb!|>! de \texttt{OCaml},
etc.

\printbibliography[heading=bibintoc]

\small
\printindex

\end{document}